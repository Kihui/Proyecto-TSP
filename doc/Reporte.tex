\documentclass[12pt]{article}
%Paquetes
\usepackage[left=2cm,right=2cm,top=3cm,bottom=3cm,letterpaper]{geometry}
\usepackage{lmodern}
\usepackage[T1]{fontenc}
\usepackage[utf8]{inputenc}
\usepackage[spanish,activeacute]{babel}
\usepackage{mathtools}
\usepackage{amssymb}
\usepackage{enumerate}
\usepackage{tabularx}
\usepackage{wasysym}
\usepackage{listings}
\usepackage{graphicx}
\usepackage{hyperref}
%\usepackage{graphicx}

%Preambulo
\title{Cómputo Evolutivo \\ Proyecto 1: Agente Viajero}
\author{Andrea Itzel González Vargas \\
  Carlos Gerardo Acosta Hernández}
\date{Facultad de Ciencias UNAM \\ Entrega: 07/09/16}
\begin{document}
\maketitle
\section{Introducción}
Como primer proyecto de la materia implementamos un Algoritmo Genético
para resolver instancias del
\textit{Problema del agente viajero} (TSP, por sus siglas en inglés),
-se consideró el caso simétrico.
Para lograrlo, utilizamos el framework de desarrollo provisto por el
ayudante de laboratorio, Roberto Monroy, en las prácticas de clase.\par
Dado que el framework provee de
una estructura modularizada para el desarrollo, el código pertinente y
referente a los principios de un AG -como los procesos de selección,
cruzamiento, mutación, así como los operadores que involucran- pueden
revisarse independientemente en la carpeta de ``sources'' (src/) del proyecto.\par
Como entrada, el programa recibe un archivo con extensión \textit{.tsp},
que representa, con cierta sintáxis especial dentro de un archivo de texto plano,
una instancia del problema del agente viajero.\par
Asimismo, en el directorio del proyecto
es posible encontrar numerosas instancias\footnote{Se pueden descargar de la página del curso \href{https://sites.google.com/site/unamfcienciascomputoevolutivo/assignments/assignment1/tsp.tar.gz}{aquí}.} del problema, entre ellas las que
fueron empleadas para el análisis de resultados -burma14, ulysses16, ulysses22, gr17 y gr21.
Todas ellas se pueden encontrar dentro del proyecto en el carpeta de datos (tsp/).\par

\end{document}
